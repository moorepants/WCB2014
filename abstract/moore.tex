\documentclass[10pt,letterpaper,notitlepage,twocolumn]{article}

\usepackage[top=0.75in, bottom=0.75in, left=0.5in, right=0.5in]{geometry}
\usepackage{times}
\usepackage{graphicx}

\title{Identification of human control during perturbed walking}
\author{
  Jason K. Moore, Sandra K. Hnat, Antonie J. van den Bogert\\
  Human Motion and Control Laboratory, Cleveland State University, Cleveland, Ohio, USA\\
  Email: j.k.moore19@csuohio.edu Web: http://hmc.csuohio.edu
}
\date{}

\begin{document}
\pagenumbering{gobble} % removes page numbers

\maketitle

\section*{Introduction}
%
This abstract introduces the results of identifying a simple linear controller
from a large set of data collected from able bodied walkers under prescribed
random longitudinal and/or lateral perturbations. The controller structure is
defined with assistive powered prosthetics application in mind. We show that
the identified controller is able to accurately generate estimated joint
torques.
%
\section*{Methods}
%
Herein we the example results are from data collected from a single subject
(age: 25, mass: 101 kg, height: 187 cm) walking at nominal speed of 1.2 m/s on
a treadmill (ForceLink R-Mill) while being longitudinally perturbed, i.e. a
random white noise with 5\% std around the nominal belt speed. We collect data
for four minutes at 100 Hz which includes about 200 steps. We compute the
ankle plantarflexion, knee flexion, and hip flexion angles, rates, and moments
using basic 2D inverse dynamics \cite{Winter}. % TODO : Add references.

We section the inverse dynamics time series into steps based on the right
foot's heel strike and interpolate 20 evenly spaced data points for each series
along the gait cycle. We assume a simple proportional derivative controller
that generates the joint torques given the joint angles and rates that fits
this form:
%
\begin{equation}
  m_m(\varphi) = m^*(\varphi) - \mathbf{K}s_m(\varphi)
\end{equation}
%
where $\varphi$ is the time in the right leg gait cycle, $m_m(\varphi)$ is a
vector of measured joint torques, $m^*(\varphi)$ are the joint torques employed
if the feedback error is zero, $\mathbf{K}$ is the gain matrix which multiplies
the error in the joint angles and rates with respect to the set point. This
form is linear in unknown parameters, $m^*(\varphi)$ and $\mathbf{K}$ and can
be solved for using linear least squares given enough measured data.
%
\section*{Results}
%
Here we show an example results from a controller structure which is limited
such that the joint torques are only generated from the error in the sensors
from that same joint. Figure \ref{fig:fit} shows an example prediction of the
measured ankle plantarflexion torque in the right leg by the identified control
model with a variance accounted for [VAF] of 76.8\%.
%
\begin{figure}
  \begin{center}
    \includegraphics[width=\columnwidth]{fig/fit.pdf}
    \label{fig:fit}
    %\caption{Needs one.}
  \end{center}
\end{figure}
%
\section*{Discussion}
%
We are able to identify various linear controllers that are able to predict the
measured joint torques with relative high accuracy in all joints that will be
likely useful in control design for power prosthetics.
\end{document}
