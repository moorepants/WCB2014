\documentclass[10pt,letterpaper,notitlepage]{article}

\usepackage[top=0.75in, bottom=0.75in, left=1.0in, right=1.0in]{geometry}
\usepackage{graphicx}
\usepackage{caption}
\usepackage{subcaption}

\title{Identification of closed looop human control during perturbed walking}
\author{
  Jason K. Moore, Sandra K. Hnat, Antonie J. van den Bogert\\
  Human Motion and Control Laboratory, Cleveland State University, Cleveland, Ohio, USA\\
  Email: j.k.moore19@csuohio.edu Web: http://hmc.csuohio.edu
}
\date{}

\begin{document}
%\pagenumbering{gobble} % removes page numbers

\maketitle

\section*{Introduction}
%
This abstract introduces example results from identifying a simple linear
controller from a large set of data collected from able bodied walkers under
prescribed random longitudinal perturbations. The controller structure is
defined with assistive powered prosthetic applications in mind. We show that
the identified controller is able to accurately generate estimated joint
torques with feedback control.
%
\section*{Methods}
%
The example results are from data collected from a single subject (age: 25,
mass: 101 kg, height: 187 cm) walking at nominal speed of 1.2 m/s on a
treadmill (ForceLink R-Mill) while being longitudinally perturbed, i.e. a
random white noise with 5\% std around the nominal belt speed. We collect data
for four minutes at 100 Hz which includes about 200 steps. We compute the ankle
plantarflexion, knee flexion, and hip flexion angles, rates, and moments using
basic 2D inverse dynamics.

We section the inverse dynamics time series into steps based on the right
foot's heel strike and interpolate 20 evenly spaced data points for each series
along the gait cycle. We assume a simple scheduled proportional derivative
controller that generates the joint torques given the joint angles and rates
that fits the following form.

\begin{equation}
  \mathbf{m}(t) = \mathbf{m}^*(\varphi(t)) -
  \mathbf{K}(\varphi(t))\mathbf{s}(t)
\end{equation}
%
\begin{figure}[b]
  \centering
  \begin{subfigure}[b]{0.5\textwidth}
    \includegraphics[width=\textwidth]{fig/gains.pdf}
    \caption{Scheduled gains for right (blue) and left (red) legs.}
    \label{fig:gull}
  \end{subfigure}%
  ~ %add desired spacing between images, e. g. ~, \quad, \qquad etc.
  %(or a blank line to force the subfigure onto a new line)
  \begin{subfigure}[b]{0.5\textwidth}
    \includegraphics[width=\textwidth]{fig/fit.pdf}
    \caption{Predicted torque against independent validation data.}
    \label{fig:tiger}
  \end{subfigure}
\end{figure}
%
where $t$ is an instance of time, $\varphi(t)$ is the phase in the right leg
gait cycle, $\mathbf{m}(t)$ is a vector of joint torques,
$\mathbf{m}^*(\varphi(t))$ is a vector of the reference joint torques,
$\mathbf{K}(\varphi(t))$ is a gain matrix scheduled with respect to gait phase
which multiplies the vector of joint angles and rates, $\mathbf{s}(t)$. This
equation is linear in the gains and the reference torques. Given sufficient
joint angle, rate, and torque measurements, the reference torques and the gains
can be solved for using linear least squares.
%
\section*{Results}
%
Here we present an example result from a controller structure which is limited
to joint torque generation only from error in the sensors from the same joint.
Figure \ref{fig:gull} shows the estimates of the scheduled gains with respect
to the percent gait cycle in each leg and Figure \ref{fig:tiger} an example
prediction of the measured ankle plantarflexion torque in the right leg by the
identified control model.
%
\section*{Discussion}
%
We are able to identify various linear controllers that are able to predict the
measured joint torques with relative high accuracy in all joints that will be
likely useful in control design for powered prosthetics.
\end{document}
